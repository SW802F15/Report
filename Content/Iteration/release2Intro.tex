With the basic music plater application completed in the 2nd iteration we outlined the configuration for the 3rd and 4th iterations in the configuration table seen in \ref{table:config1} taken from \citet{essence:config}.


\begin{table}
\begin{tabular}{l|l|l|l|l}
         & \textbf{Paradigm} & \textbf{Product} & \textbf{Project} & \textbf{Process} \\ \hline
\parbox[t][4cm][c]{0.02\textwidth}{\rotatebox{90}{Focus}} %TODO find vertical height
	& \parbox[t]{0.20\textwidth}{\small 
		\textit{Reflection} \newline
		Challenge: \newline
		Can we improve the running experience. \newline
		Use context: \newline
		Running while listening to music from a smartphone.
	}
	& \parbox[t]{0.20\textwidth}{\small 
		\textit{Affordance} \newline
		Running Pacer. \newline
		Option: \newline
		Interval Trainer / Training programs. \newline
		Step Counter (no music). \newline
		Music Player (no running). \newline
	}
	& \parbox[t]{0.20\textwidth}{\small  
		\textit{Vision} \newline
		Vision: \newline
		Running Pacer by use of music. \newline
		Use step counter to match song to running tempo.
	}
    & \parbox[t]{0.20\textwidth}{\small 
	    \textit{Facilitation} \newline
	     Focus on immediate benefits to user. %todo what?
   } \\ \hline
\parbox[t][5cm][c]{0.02\textwidth}{\rotatebox{90}{Overview}}
	& \parbox[t]{0.20\textwidth}{\small 
    \textit{Stakeholders} \newline
    Runner.
	}
	& \parbox[t]{0.20\textwidth}{\small 
		\textit{Design} \newline
	 	Running Pacer with music player-like design and functionalities.  
	}
	& \parbox[t]{0.20\textwidth}{\tiny
		\textit{\small  Elements} \newline
		Grounds: \newline
		Paced music gives better running experience. \newline
		Warrant: \newline
		When running, it is human nature to match pace with the music playing. \newline
		Qualifier: \newline
		Detailed music information (bpm) required. Precise SPM measurement required. \newline
		Rebuttal: \newline
		Use context: \newline
		To match songs with pace. \newline
		Limitations: \newline
		Step counter measures step frequency.
	}
	& \parbox[t]{0.20\textwidth}{\small 
		\textit{Evaluation} \newline
		Procedure: \newline Iteration retrospective by surrogate customer \newline
		Criteria: \newline Evaluate immediate functionality based on acceptance tests.
	}\\ \hline
\parbox[t][3.5cm][c]{0.02\textwidth}{\rotatebox{90}{Details}}
	& \parbox[t]{0.20\textwidth}{\small 
		\textit{Scenarios}\newline
		Automatically fades into songs, which fit running pace.\newline
		Use private collection of MP3 files as a basis for exercise/running.
	}
	& \parbox[t]{0.20\textwidth}{\small 
		\textit{Components}\newline
		Music player. \newline
		Music library. \newline
		Step Counter.
		
	}
	& \parbox[t]{0.20\textwidth}{\small 
		\textit{Features}\newline
		Running pacer.\newline
		Music player.\newline
		Step counting.
	}
	& \parbox[t]{0.20\textwidth}{\small 
		\textit{Findings} \newline
		Extracting beat pr minute information from MP3 files is not practical. \newline
		Controlling the device while running can be difficult.
	}\\ \hline     
\end{tabular}
\caption[Table caption text]{Table taken from }
\label{table:config1}
\end{table}

\begin{itemize}
\item Paradigm Focus: The challenge is to improve running experience by matching music beat to pace. The application is supposed to be used while running.
\item Paradigm Overview: Stakeholder is the user who is using the application.
\item Paradigm Details: For this release we have a two main scenarios to fulfil. First we want the application to automatically select appropriate (songs with a matching beat) songs to play as the user runs at varying paces. The song pool to select from should be customisable for each device.
\item Product Focus: The application is first and foremost a running pace, meaning the purpose is to match music beat with steps per minute (pace). It is possible to add an option for interval training so the application no longer adapt music beat based on pace, but encourage the user to adapts his or her pace to predetermined intervals of low and high beat music. Additional training programs could also be added. Further more the application can also server as a step counter not playing any music just recording steps taken and displays steps per minute, and the other way around just playing music without taking steps per second into account.
\item Product Overview: The application design should be based on a ``familiar music player design'', it should have the expected functionally from a music player and new elements (such as steps per minute) should expand on the familiar design. 
\Kristian{consider ``familiar music player design''}
\item Product Details: This configuration consists of 3 components the music player, a library and a step counter. The music player is used to play songs and serve as the primary interface for the user. The music library is one or more folders with MP3 files on the device. The step counter makes use of the device's accelerometer to register steps taken.
\item Project Focus: The overall vision of the application is as a running pacer which matches the beat of songs to played with running pace.
\item Project Overview: Grounds are described in \cref{chap:intro}.
\item Project Details: \Kristian{this seems similar to \textit{Affordance}}
\item Process Focus: The evaluation is based on benefits to the user(runner).
\item Process Overview: We evaluate the application as a surrogate customer at iteration end/review. The evaluation is based on the fulfilment of acceptance tests.
\item Process Details: 

\end{itemize}