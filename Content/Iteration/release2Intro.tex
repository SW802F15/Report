With the core music player application completed in the 2nd iteration we outlined the configuration for the 3rd and 4th iterations in the configuration table seen in \Cref{table:config1}, which is based on \texttt{Essence: Pragmatic Software Innovation} \citep[ch. 3]{essence:config}. The figure should be read vertically, and is meant to provide an overview; each field is explained in more detail in this section. This approach presents the different aspects of the release, including what problem the application should solve and how, and it provides a basis for further development. The configuration table outlines the planned release, and not the finished release.


\begin{table}
\begin{tabular}{l|l|l|l|l}
         & \textbf{Paradigm} & \textbf{Product} & \textbf{Project} & \textbf{Process} \\ \hline
\parbox[t][4cm][c]{0.02\textwidth}{\rotatebox{90}{Focus}} %TODO find vertical height
	& \parbox[t]{0.20\textwidth}{\small 
		\textit{Reflection} \newline
		Challenge: \newline
		Can we improve the running experience. \newline
		Use context: \newline
		Running while listening to music from a smartphone.
	}
	& \parbox[t]{0.20\textwidth}{\small 
		\textit{Affordance} \newline
		Running Pacer. \newline
		Option: \newline
		Interval Trainer / Training programs. \newline
		Step Counter (no music). \newline
		Music Player (no running). \newline
	}
	& \parbox[t]{0.20\textwidth}{\small  
		\textit{Vision} \newline
		Vision: \newline
		Running Pacer by use of music. \newline
		Use step counter to match song to running tempo.
	}
    & \parbox[t]{0.20\textwidth}{\small 
	    \textit{Facilitation} \newline
	     Focus on immediate benefits to user. %todo what?
   } \\ \hline
\parbox[t][5cm][c]{0.02\textwidth}{\rotatebox{90}{Overview}}
	& \parbox[t]{0.20\textwidth}{\small 
    \textit{Stakeholders} \newline
    Runner.
	}
	& \parbox[t]{0.20\textwidth}{\small 
		\textit{Design} \newline
	 	Running Pacer with music player-like design and functionalities.  
	}
	& \parbox[t]{0.20\textwidth}{\tiny
		\textit{\small  Elements} \newline
		Grounds: \newline
		Paced music gives better running experience. \newline
		Warrant: \newline
		When running, it is human nature to match pace with the music playing. \newline
		Qualifier: \newline
		Detailed music information (bpm) required. Precise SPM measurement required. \newline
		Rebuttal: \newline
		Use context: \newline
		To match songs with pace. \newline
		Limitations: \newline
		Step counter measures step frequency.
	}
	& \parbox[t]{0.20\textwidth}{\small 
		\textit{Evaluation} \newline
		Procedure: \newline Iteration retrospective by surrogate customer \newline
		Criteria: \newline Evaluate immediate functionality based on acceptance tests.
	}\\ \hline
\parbox[t][3.5cm][c]{0.02\textwidth}{\rotatebox{90}{Details}}
	& \parbox[t]{0.20\textwidth}{\small 
		\textit{Scenarios}\newline
		Automatically fades into songs, which fit running pace.\newline
		Use private collection of MP3 files as a basis for exercise/running.
	}
	& \parbox[t]{0.20\textwidth}{\small 
		\textit{Components}\newline
		Music player. \newline
		Music library. \newline
		Step Counter.
		
	}
	& \parbox[t]{0.20\textwidth}{\small 
		\textit{Features}\newline
		Running pacer.\newline
		Music player.\newline
		Step counting.
	}
	& \parbox[t]{0.20\textwidth}{\small 
		\textit{Findings} \newline
		Extracting beat pr minute information from MP3 files is not practical. \newline
		Controlling the device while running can be difficult.
	}\\ \hline     
\end{tabular}
\caption[Table caption text]{Table taken from }
\label{table:config1}
\end{table}

\begin{itemize}
\item \textbf{Paradigm Focus}: The challenge is to improve the running experience by matching the tempo of the music with the pace of the runner. The application is supposed to be used while running.

\item \textbf{Paradigm Overview}: Stakeholder is the user who is using the application, i.e., the runner.

\item \textbf{Paradigm Details}: For this release there are two main scenarios to fulfil. Firstly the application should automatically select appropriate songs (songs with a matching BPM) to play as the user runs at varying paces. Secondly it should be possible to customise the song pool from which songs are selected.

\item \textbf{Product Focus}: The application is first and foremost a running pacer, meaning the purpose is to match music tempo with the user's pace, measured in steps per minute. It is possible to add an option for interval training so the application no longer adapt the music's tempo based on pace, but encourage the user to adapt his or her pace to predetermined intervals of low and high tempo music. Additional training programs could also be added. The application can also serve as a step counter, not playing any music but just recording steps taken and displaying it as either total steps or steps per minute. Similarly it can just play music without taking SPM into account.

\item \textbf{Product Overview}: The application design should be based on a familiar music player design, and it should have the expected functionally from a music player, but new elements (such as SPM) should expand on the familiar design. The navigation buttons should be placed, so they are easy to use while moving.

\item \textbf{Product Details}: This configuration consists of 3 components: the music player, a music library, and a step counter. The music player is used to play songs and serve as the primary interface for the user. The music library is one or more folders with music files on the device. The step counter makes use of the device's accelerometer to register steps taken.

\item \textbf{Project Focus}: The overall vision of the application is as a running pacer which matches music to the user's running pace.

\item \textbf{Project Overview}: \newline
\textit{Grounds:} As described in \cref{chap:intro}, music can have an influence on the running experience. Having music that matches your running pace makes sure it does not have a negative impact on the exercise session, and having the application find appropriate music automatically makes the process of finding good running music less tedious. \newline
\textit{Warrant}: Many people prefer running while listening to music, and it will often result in people trying to match their running pace with the music playing. \newline
\textit{Qualifier}: It is important that the song playing always, or as much as possible, matches the user's pace, so their running experience is not disrupted. \newline
\textit{Rebuttal}: The user may change pace in the middle of a song, and this poses a problem. It could be solved by altering the music's tempo just enough to match the new pace, without distorting the sound of the music too much. Another solution could be to transition to another piece of music, when the new pace is confirmed stable.

\item \textbf{Project Details}: The application should work as a running pacer, a music player, and a step counter.

\item \textbf{Process Focus}: Through an iterative process there will be focus on the most important features as decided by the customer, providing immediate benefits to the user.

\item \textbf{Process Overview}: We evaluate the application from the customer's point of view at the end of each iteration. The evaluation is based on the fulfilment of acceptance tests.

\item \textbf{Process Details}: While running it can be difficult to navigate the application, as there is limited feedback when using a touch screen. Instead it should be possible to use gestures or taps to control the application, giving it a more 
consistent user experience.
\end{itemize}