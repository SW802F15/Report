\subsection{1st Iteration}
In the 1st iteration we had to get used to using the XP practices. First of all we needed to prepare issues for the iteration, so we sat down and pretended to be both programmers and customer, talked about some user stories, wrote them down the best we could. We looked at the user stories, turned them into issues, and started a game of planning poker: estimating and prioritising each issue. We also set a goal, indicating how many issues we expected to finish during the iteration, but that was a complete shot in the dark, given the fact that it was the first iteration with the team and we had no knowledge of which code velocity to expect, i.e., how many work effective work hours we would be able to put into an iteration.

Then it was time to sit down and get working, and the goal was to use the XP practices as much as possible, and that included pair programming. So when it was time to solve an issue, one person sat down as the driver, and the other joined in, either by using screen sharing software or looking at the same monitor. At this point we only had access to laptop monitors, resulting in a poor experience for the co-driver, as several of the laptops in use have monitors that only show content properly if you are placed directly in front of them.
This lead to pair programming being more difficult to execute, and we decided to get a proper monitor for each pair, to make sure the work environment would not have a negative impact on the practice.

Some times we found that pair programming was a hindrance for our work, and this was especially the case when we had to solve tasks that require a lot of experimentation due to our lack of knowledge of the subject. In these cases, we would let one person do the experimentation, leading to another person being forced to work alone as well.

All in all, in this stage of the project practices were still being adapted, and that lead to certain aspects of the process still being relatively unorganised. Similarly, we were under the impression that refactoring was something we did not have to think about yet -- so we did not. It gave us one less thing to think about, but it also resulted in us leaving bad smells behind, that could have been caught early.

At the end of the iteration we evaluated our planning, and we found that our estimation of trivial issues (i.e., issues we already knew how to solve and that did not need more than a couple of hours) were fine. On more complex issues, on the other hand, we had trouble, and sometimes we had spend twice as much time as anticipated. This was often the case because the issues required a lot of research or simply were more complex than expected, but on the bright side it gave us some experience we could use for planning the next iterations. We also noted that we worked more hours than expected, but this was not a surprise, as our expectations were set purely based on feeling with no previous experience.


%\begin{itemize}
%	\item Playing music files located on the device.
%	\begin{itemize}
%		\item Stop/Pause functionality.
%		\item Playback should continue when screen is turned off.
%	\end{itemize}
%	\item Selected music files to play based on beat per minute.
%	\begin{itemize}
%		\item Replay previously played songs and skip a song and select a new.
%	\end{itemize}
%\end{itemize}

