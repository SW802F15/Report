\subsection{Test Goal and Setup}\Christoffer{Should this title be changed?}
% why did we perform the tests?
The tests of the accelerometer was performed mainly to determine an appropriate way to detect steps. There were no libraries available for the Android API version 16, which the devices tested used, so we had to develop our own way of determining when a step was taken.

% how did we perform the tests
Four different people performed each test in order to get a more general way to detect steps. If only one person was tested, the step counting might not be accurate for another. While testing four different does not guarantee that it works for everyone, it was reasoned that it did not hurt the general precision compared to only testing on one person. 

Two different smartphones were used for the tests: Samsung Galaxy Note 3 and the Nexus S3. The accelerometers were tested with four different motion styles\Christoffer{bedre ord? running styles virkede ikke passende når gågang er en af vores ``styles''}: sprint, run, jog and walk. In addition to this, different phone positions were used for each of the four styles. The positions were: While the phone is in the tester's hand, pocket or on their arm.
Additionally the phones' accelerometers were also tested when there were only being held in a tester's hand and lying still on a table. These were only performed for one tester since they do not vary between people as running or jogging styles do.

\subsection{Test Results}
% what did we measure
Different data was collected in the accelerometer tests. X, Y and Z axis information was collected. Additionally, the time between each reading of the axes was measured and each tester was asked to count their steps in every test where the device was in motion. 

% what was the results of our tests