\section{Acceptance Test}
As a way of making sure that a user story is fulfilled an acceptance test should be written. An acceptance test describes how the application should perform in a given situation for the user story to be complete, and the progress of the project can be measured by the amount of passing acceptance tests.

The acceptance tests should be written by the customers, possibly with some help from a dedicated tester, to make sure the program does exactly what they expect it to do. The acceptance tests should preferably be automated, and if necessary they can be implemented by a programmer. The idea of making them automated is to make it possible to use them as regression tests, making sure that refactoring and/or changes to other parts of the code do not break the functionality of features that have already been implemented.
\Dan{Should maybe have some sources?}

The important thing when writing a test is to make sure it is completely unambiguous and reproducable. An example of a user story and the associated acceptance test could be:

\story{Keep playing when screen is off.}{High}
{As a user, I, would like to save power by turning off the screen while the music is playing.}
{\begin{itemize}
\item The music keeps playing after the \texttt{Power} button is pressed.
\end{itemize}}
{\begin{itemize}
\item Precondition: 
\subitem $-$ The application must be launched in the MusicPlayer Activity.

\item Procedure:
\subitem $-$ Press \texttt{Play} button.
\subitem $-$ Verify music is playing.
\subitem $-$ Press \texttt{Power} button.
\subitem $-$ Verify screen turns off.

\item Postcondition:
\subitem $-$ Verify the music is still playing.
\end{itemize}}

\Dan{Should we write more about how we wrote this test?}


%This section should describe the specification, process, implementation, and results of the acceptance test.

%Maybe mention that it is done in collaboration with the TOV course mini project.