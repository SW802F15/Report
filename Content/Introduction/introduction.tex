%First Problem - Idea Domain%
Running is a popular form of exercise, however, it can be a tedious and uninspiring endeavour.
To improve the experience, \citet{musicRunEffectArticle} found that 
\textit{``... participants enjoyed what they were doing [running] more when they were listening to music of any sort when compared to when they were not.''}

It was further concluded by \citet{musicRunEffectArticle} that the volume and tempo of the music influenced the running experience.
They concluded that the running pace for novice runners, while listening to relatively low-tempo music, was slower than when not listening to music. Additionally listening to high-tempo music resulted in a faster running pace, compared to when not listening to music.

This conclusion is in disagreement with the conclusion of \citet{musicNoPerformanceEffect} which suggests, that \textit{``... music had no impact on mean power output.''}. \citet{musicNoPerformanceEffect} measure the running pace by mean power output, nevertheless, they did not see any impact on the running pace by listening to music.

As a result, we can not definitively conclude whether music of different tempo will affect the running experience differently. 
However by adhering to \citet{musicRunEffectArticle}'s conclusion, we can only improve the running experience, since \citet{musicNoPerformanceEffect} concludes there can be no negative impact, by playing music.

Today many runners use their smartphone as a music player, which can either be placed in their hand, pocket, or on their arm.
The sensors in a smartphone enable monitoring the pace and speed of the runner.
Based on this knowledge the first problem can be stated:

\begin{center}
\textit{How can we provide music with an appropriate tempo, compared to the current pace, to the runner through the use of a smartphone?}
\end{center}

%Second Problem - UI%
\noindent Operating a smartphone while running is difficult, especially if it is placed in the pocket or on the arm of the runner.
In order for the runner to operate the smartphone properly, the runner would have to stop running, or focus more than normally, which can disrupt the runner's form, and lead to injuries and accidents.
Based on this knowledge the second problem can be stated: 

\begin{center}
\textit{How can a smartphone application be operated without disrupting the runner's form and/or concentration?}
\end{center}\Alexander{Define what ``operated'' means, what operations?}

\Alexander{Make sure the third problem is the first thing written on page 2}
%Third Problem - Methodology%
%\noindent Most large scale software projects fail. 
\noindent According to \citet{gartner:failure} \textit{``A full 66 percent of large scale projects fail ...”}, and although this is not a large scale project, some of the same pitfalls exist.
One way to avoid some of these pitfalls, is \textit{``... using a structured systems development methodology ...”}, since it according to \citet{dorsey:methodologyReason} \textit{``... is one of the critical success factors in a systems development project.”}.

We will in this project focus on the development methodology \texttt{Extreme Programming} (XP).
XP is used because its an interesting methodology and it is development oriented.
Furthermore, XP's requirements of self-organising teams, iteration length, and team size fit well with this project. 
Based on this knowledge the third problem can be stated:

\begin{center}
	\textit{How do we adapt the structured systems development methodology, Extreme Programming, to our project?}
\end{center}