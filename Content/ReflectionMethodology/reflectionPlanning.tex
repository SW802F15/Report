\section{Planning}
Planning poker was used for planning. We found planning poker made sure that everybody was heard in regards to planning. It also made sure that we had a common understanding of the task to be planned, e.g. if there was a big difference in the estimates, the group might not agree on what the task entails.

At first, we did planning but we did not prioritise the tasks. This was a mistake because the most important task was not always the first to be finished. We tried different ways of timing the solving of tasks. First, we just looked back in the calendar for the iteration and looked at which days we worked on what. Then we made a rough estimated of how long it took to solve the task. 

We experienced that it was difficult to get meaningful timing data this way. Later, a tool was found to help us track time. Based on this, we could determine how good we were at estimating tasks in relation to how long it took to solve them. Additionally, we could keep track of how many development hours were used each iteration. 

A disadvantage of using the tool was that since much weight was put on detailed timing, the tool itself generated overhead. We later moved on to just estimating units.

As a result of our estimate reviews, we found a recurring pattern in our estimates. Trivial tasks were generally easy to estimate accurately. Smaller, one unit, tasks were more likely to exceed their estimate. Lastly, the big tasks, four plus units, were likely to be over-estimated with the exception of a few which grew bigger. We suspect this was because we should have decomposed the tasks further.

Having big tasks with no specific goals, such as ongoing refactoring, did not work well, because they made us unproductive. Refactoring for example was later decomposed into concrete tasks which worked better.

\Dan{Mention somewhere: Fragmentation of time.}