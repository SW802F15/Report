\section{Planning Game}
Planning is a useful tool.
We used it primarily as a tool for measuring our velocity, to get an overview of our progress, and to know if we were falling behind schedule.
Additionally it helps keep the project progressing as the customer desires, by allowing the customer to continuously create user stories, which should be implemented.

\subsection{The Customer's Role}
The main responsibilities of the customer is to create user stories and prioritise in which order they should be implemented.
As stated in \Cref{}, we use a 
%User Stories
%Prioritising

\subsection{Making Estimations}
To properly plan an iteration, it is important to have precise estimations of the issues at hand.
It allows the customer to easier prioritise each issue or story, and it allows the developers to have an idea of how many issues they can finish during the iteration.

Estimating an issue correctly is, however, a difficult task.
Many variables play a role in how long an issue takes to solve, and it is often impossible to foresee them all.
To try to reduce this problem we employed the practice of planning poker, where each member of the team gives their own (anonymous) estimation, and the final estimation is discussed afterwards.
We found that in the first iterations we underestimated the amount of time needed for the more difficult issues, while we generally estimated correctly when it came to trivial issues.

\subsection{Code Velocity}
%Velocity and fragmentation

\subsection{Summary}
%Overhead
%Flexibility (in terms of changes)
