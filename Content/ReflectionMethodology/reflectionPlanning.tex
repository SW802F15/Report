\section{Planning Game}
Planning is a useful tool.
We used it primarily as a tool for measuring our velocity, to get an overview of our progress, and to know if we were falling behind schedule.
Additionally it helps keep the project progressing as the customer desires, by allowing the customer to continuously create user stories, which should be implemented.

\subsection{The Customer's Role}
The main responsibilities of the customer, in regards to planning, is to create user stories and prioritise in which order they should be implemented.
As stated in \Cref{def:onsitecustomer}, we used a surrogate customer, which meant we were responsible for the tasks normally appointed to the customer.

%User Stories
The first task we needed to do, was create the user stories.
The creation of the user stories went fairly easy, but we later found that these were much larger than desired.
We would therefore split them into several smaller, more well defined, user stories.
We felt that having an oral agreement was a more flexible tool for creating issues, so a lot of the time the stores were not written down.
Taking the time to write the issues down would have helped us make sure, that there was total agreement about what each story should achieve, but it would also result in having to spend more time on rewriting the story if changes occurred, possibly discouraging us from making changes.

%Prioritising
After the user stories were estimated it was our responsibility, as surrogate customers, to prioritise in what order the user stories should be implemented.
Usually we prioritised them as we suspected an on-site customer would, according to cost/benefit.
However, after the basic modules of the program were implemented, we started letting our developer hearts influence the decisions.
We started to prioritise the user stories by novelty, which essentially means by what we found new and interesting to implement.
This caused the product to be more of a proof-of-concept product instead of the initial intended usable product.
Although the product did not end up as what was intended from the get go, it showed us that XP was capable of embracing change.


\subsection{Making Estimations}
To properly plan an iteration, it is important to have precise estimations of the issues at hand.
It allows the customer to easier prioritise each issue or story, and it allows the developers to have an idea of how many issues they can finish during the iteration.

Estimating an issue correctly is, however, a difficult task.
Many variables play a role in how long an issue takes to solve, and it is often impossible to foresee them all.
To try to reduce this problem we employed the practice of planning poker, where each member of the team gives their own (anonymous) estimation, and the final estimation is discussed afterwards.
We found that in the first iterations we underestimated the amount of time needed for the more difficult issues, while we generally estimated correctly when it came to trivial issues.

\subsection{Code Velocity}
%Velocity and fragmentation

\subsection{Summary}
%Overhead
%Flexibility (in terms of changes)
