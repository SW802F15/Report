The non-graphical user interface (NGUI) provides a way of using the application without having to look at the screen. As mentioned in \Cref{chap:intro}, this is a required feature as the user is likely not able to look at the screen while running, but they still need to operate the application.

Several approaches can be taken when it comes to creating a NGUI, e.g. an interface which is controlled by voice, taps on the screen, or gestures, can be implemented. Taps were chosen over gestures, or voice control as they are easier to perform while running. 

%\Kristian{repeat?}
%The chapter contains a description of the user stories created, which acted as a basis for the development of the NGUI in regards to the requirements it should fulfill. Points of interest from its implementation are also described.