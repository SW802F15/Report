\section{Problem Statement}
Running is a popular form of exercise, however it can be a tedious and boring endeavour.
To better the experience, \citet{musicRunEffectArticle} found that 
\textit{``... participants enjoyed what they were doing [running] more when they were listening to music of any sort when compared to when they were not.''}

It was further discovered by \citet{musicRunEffectArticle} that the volume and tempo of the music influenced the running experience.
They discovered that the running speed, while listening to relatively slow music, was slower than when not listening to music. Additionally listening to fast-paced music resulted in a faster running speed, compared to when not listening to music.

%%%%% BELOW SHOULD BE WRITTEN BETTER  %%%%%
Today many runners use their smartphone as a music player, which can either be placed on the arm or in their pocket.
The sensors in a smartphone allows for reading the steps per minute and speed of the runner. This can be used to calculate and manage suitable music for the runner at a given time and speed.


