\section{Problem Statement}
Running is a popular form of exercise, however it can be a tedious and uninspiring endeavour.
To improve the experience, \citet{musicRunEffectArticle} found that 
\textit{``... participants enjoyed what they were doing [running] more when they were listening to music of any sort when compared to when they were not.''}

It was further discovered by \citet{musicRunEffectArticle} that the volume and tempo of the music influenced the running experience.
They discovered that the running speed, while listening to relatively slow music, was slower than when not listening to music. Additionally listening to fast-paced music resulted in a faster running speed, compared to when not listening to music.

Today many runners use their smartphone as a music player, which can either be placed on their arm or in their pocket.
The sensors in a smartphone allows for reading the steps per minute and speed of the runner.

\begin{center}
\textit{How can this data be used and calculated upon to provide the runner with an enjoyable running experience?}
\end{center}

\Ivan{Will the app follow the users tempo or will the user follow the music?}
\Ivan{Think about and discuss difference types of running e.g. interval.}
\Ivan{Maybe have options to choose between 3 mode: plain (music follows user), fast (user follows music), interval or other template (user follows music of different BPM).}

With a smartphone placed on their arm or in their pocket, it is difficult to navigate applications through visual feedback. Furthermore operating the smartphone while running decreases the navigability of the smartphone. \Alexander{SOURCE} 
The runner's form should not be disrupted by navigating. 
\Alexander{Re-write last (previous $\uparrow$) line.}

\begin{center}
\textit{How can a smartphone application be operated without disrupting the runner's form?}
\end{center}