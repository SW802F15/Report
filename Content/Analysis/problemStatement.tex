\section{Problem Statement}
Running is a popular form of exercise, however, it can be a tedious and uninspiring endeavour.
To improve the experience, \citet{musicRunEffectArticle} found that 
\textit{``... participants enjoyed what they were doing [running] more when they were listening to music of any sort when compared to when they were not.''}

It was further concluded by \citet{musicRunEffectArticle} that the volume and tempo of the music influenced the running experience.
They concluded that the running pace for novice runners, while listening to relatively low-tempo music, was slower than when not listening to music. Additionally listening to high-tempo music resulted in a faster running pace, compared to when not listening to music.

This conclusion is in disagreement with the conclusion of \citet{musicNoPerformanceEffect} which suggests, that \textit{``... music had no impact on mean power output''}.

As a result, we can not definitively conclude whether music of different tempo will affect the running experience differently. 
However by adhering to \citet{musicRunEffectArticle}'s conclusion, we can only improve the running experience, since \citet{musicNoPerformanceEffect} concludes there can be no negative impact, by playing music.

Today many runners use their smartphone as a music player, which can either be placed in their hand, pocket, or on their arm.
The sensors in a smartphone enable monitoring the pace and speed of the runner.
From this knowledge the first problem can be stated:

\begin{center}
\textit{How can we provide music with an appropriate tempo, compared to the current pace, to the runner through the use of a smartphone?}
\end{center}

%Second Problem - UI%
Operating a smartphone while running is difficult, especially if it is placed in the pocket or on the arm of the runner.
In order for the runner to operate the smartphone properly, the runner would have to stop running, or focus more than normally, which can disrupt the runner's form, which can lead to injuries and accidents.

From this knowledge the second problem can be stated: 

\begin{center}
\textit{How can a smartphone application be operated without disrupting the runner's form or concentration?}
\end{center}

%\pagebreak
%\Ivan{First problem:}
%\Ivan{Will the app follow the users tempo or will the user follow the music?}
%\Alexander{It is now clear that the application should follow the pace of the user.}
%\Ivan{Think about and discuss difference types of running e.g. interval.}
%\Alexander{Upon reflection we have decided to `move' the training program problem, since it is very trivial when the stated problem is solved.}
%\Ivan{Maybe have options to choose between 3 mode: plain (music follows user), fast (user follows music), interval or other template (user follows music of different BPM).}
%\Alexander{We have chosen not to focus on these given controversy, lack of complexity and interest.}

%\Ivan{Second problem:}


%\Alexander{General notes:}
\Alexander{BPM error deviation = 7.5 BPM - \url{http://www.cs.virginia.edu/~stankovic/psfiles/MusicalHeart-SenSys2012-CameraReady.pdf}}



\Ivan{Write about XP}
\Ivan{Deliberate on what \citet{musicNoPerformanceEffect} is about.}