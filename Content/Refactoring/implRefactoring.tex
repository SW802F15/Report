\section{Implementation of Refactoring}
After a review of the project's source code, we created a list of the bad smells identified. Of these, the three most frequently occurring were
\begin{itemize}
\item Duplicated code
\item Long method
\item Inappropriate intimacy
\end{itemize}

In the following subsections, concrete examples of how we dealt with them are illustrated. 
\subsection{Refactoring of Duplicated code}
The method in \Cref{lst:dupBefore} shows a method which suffered from the duplicated code smell. As can be seen in the sample, the two for-each loops perform identical operations on a list. They both iterate over a list and if an element from that list is contained in another list, it is removed from that list.

\begin{code}{lst:dupBefore}{Example of code before refactoring.}
\begin{lstlisting}
 public List<Song> getMatchingSongs(int num, int thresholdBMP){
    ...

    for (Song song : _prevSongs){
        if(songs.contains(song)){
            songs.remove(song);
        }
    }

    for (Song song : _nextSongs){
        if(songs.contains(song)){
            songs.remove(song);
        }
    }
    ...
\end{lstlisting}
\end{code}

\Cref{lst:dup1After} shows the method after method extraction has been performed on it, and two method calls placed in place of the duplicated code.
% % % % % % % % % % % % % % % % % % % % % %

\begin{code}{lst:dup1After}{Example of code after refactoring.}
\begin{lstlisting}
 public List<Song> getMatchingSongs(int num, int thresholdBMP){
    ..
    removeDuplicateSongs(_prevSongs, songs);
    removeDuplicateSongs(_nextSongs, songs);
    ...
\end{lstlisting}
\end{code}

% % % % % % % % % % % % % % % % % % % % %

\begin{code}{lst:dup2After}{Example of extracted method from refactored code.}
\begin{lstlisting}
private void removeDuplicateSongs(List<Song> songList, List<Song> songs){
    for (Song song : songList){
        if(songs.contains(song)){
            songs.remove(song);
        }
    }
}
\end{lstlisting}
\end{code}

 \Cref{lst:dup2After} shows the extracted method, which performs the same operation as the two identical for-each loops did.
 
 As can be seen, the refactored method is both shorter, and more readable because it not has a method name which explains exactly what the for-each loop does. Thus, both readability and maintainability has been improved.
\subsection{Refactoring of Long Method}
%before 
%after
\subsection{Refactoring of Inappropriate Intimacy}
%before 
%after