\section{Android Terminology}
When explaining the implementation of the modules, some Android specific terms will be used, so a very short description of the terms needed to understand the explanations are listed here. These explanations are based on the Android API Guide Glossary \cite{android:terms}, the Android Fragments guide \cite{android:fragment} and the Sensor Overview \cite{android:sensor}.

\subsection*{Activity}
A screen in the application implemented as a Java class with layout defined as XML. It can handle UI events and call methods.

\subsection*{Fragment}
A fragment is intended to handle a part of the behaviour and UI of an Activity, especially useful for creating multi-pane UI's and adjusting the app to run on devices with different screen sizes.

\subsection*{Service}
A service runs in the background with the purpose of handling long-running, persistent actions, e.g. playing music when the app is minimised. It does not provide a user interface.

\subsection*{Sensors and Listeners}
The phone has a number of sensors, e.g. an accelerometer, and these can be accessed by listeners. A listener implements methods like \texttt{onSensorChanged()}, which is called with a value when the sensor receives a new input. Listeners are also used to detect input events like button clicks and touches.