\section{Android Terminology}
Here are some Android specific terms that will be used in the implementation sections. These explanations are based on the Android API Guide Glossary \citep{android:terms} and the Sensor Overview \citep{android:sensor}.

\subsection*{Activity}
A screen in the application responsible for handles UI events. It is implemented as a Java class with a layout primarily defined in XML.

%\subsection*{Fragment}
%A fragment is intended to handle a part of the behaviour and UI of an Activity, especially useful for creating multi-pane UI's and adjusting the application to run on devices with different screen sizes.
%\Kristian{No one writes about fragments anyway}

\subsection*{Service}
A service runs in the background with the purpose of handling long-running, persistent actions, e.g. playing music when the application: is minimised. It does not provide a user interface. \Alexander{Look at sentence structure, is ``application: is minimised'' the intended structure?}

\subsection*{Sensors and Listeners}
The device can have a number of sensors, e.g. an accelerometer, and these can be accessed by listeners. A listener implements methods like \texttt{onSensorChanged()}, which is called with a value when the sensor receives a new input. Listeners are also used to detect input events like button clicks and touches.
