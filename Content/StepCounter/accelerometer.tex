\subsection{Accelerometer}
There were no libraries available for the Android API version 16, so we had to develop a way of determining when a step was taken based on the data from the accelerometer.

\subsubsection{Test Goal}
% why did we perform the tests?
The tests of the accelerometer was performed to determine behaviour during different movement patterns. 

\subsubsection{Test Setup}
% how did we perform the tests
For the tests we had:
\begin{itemize}
\item Four people
\item Two smartphone (one Samsung Galaxy Note II and one Samsung Galaxy S III)
\end{itemize}

We use four different people to ensure the measurement are general.
Although four people does not necessarily represent all people, but it is better than one.
\Alexander{Better reasoning $\uparrow$}

We use two different types of smartphones to ensure consistency.
\Alexander{Describe reasoning $\uparrow$}

\subsubsection{Test Procedure} \Alexander{Should be rewritten}
In order to determine the behaviour of the accelerometer in different movement patterns, we devised five tests.
The data we will gather from each test is measurements of:
\begin{itemize}
\item Movement according to X, Y, Z axis from the accelerometer.
\item Position according to X, Y, Z axis from the gyroscope.
\item Gravity.
\item Estimated number of steps.
\item Time between measurements.
\end{itemize}

\paragraph{The first test} consisted of each person placing both phones in each pocket and then performing the following tasks.

The test is iterated four times, one for each possible rotation of the phone.
For each iteration the following tasks are performed.

\begin{enumerate}
\item Walk 150 steps, save the gathered data.
\item Jog 150 steps, save the gathered data.
\item Run 150 steps, save the gathered data.
\item Sprint 150 steps, save the gathered data.
\end{enumerate}

\paragraph{The second test} consisted of each person placing both phones on their arms and then performing the same tasks as in the first test.

\paragraph{The third test} consisted of each person placing both phones in their hands and then performing the same tasks as in the first test.

\paragraph{The fourth test} consisted of both smartphones being placed statically on a level table.

\paragraph{The fifth test} consisted of both smartphones being held statically by a person.


\subsubsection{Test Results}
% what did we measure


% what was the results of our tests

\Alexander{Check for correct tense (past-tense vs. present-tense) in entire chapter.}