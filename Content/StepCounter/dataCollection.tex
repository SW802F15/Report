\subsection{Data Collection}
The first step of data collection is to figure out what data to collect, to do this we take a look at the available sensors on a android smartphones see \Kristian{ref til android side om sensors} for description. We decided to gather data from three sensors accelerometer, gravity and gyroscope we also measured the time delay between each sensor reading.


\subsubsection{Collection Procedure}
To generate the sample data we a person perform the follow tasks while carrying the smartphone in a specific position (in hand, in pocket or strapped to arm).
\begin{enumerate}
\item Walk 100 steps.
\item Jog 100 steps.
\item Run 100 steps.
\item Sprint 100 steps.
\item Alternate walk, jog, run and sprint for 100 steps.
\item Standing still for 2 mins.
\end{enumerate}

These tasks can then be repeated with for all carrying positions, different smartphones and multiple persons. In practice the tasks where carried out with two people, the smartphone Samsung Galaxy S III and in one carrying position \textit{strapped to arm}.

After data collection we are now ready to test algorithms against each other.


\subsubsection{Test Results}
% what did we measure


% what was the results of our tests

\Alexander{Check for correct tense (past-tense vs. present-tense) in entire chapter.}