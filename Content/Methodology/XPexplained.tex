\section{Extreme Programming explained}
% 2 XP's: 2000 and 200? - differences.
% General introduction to XP2k
Extreme Programming (XP) is a software development methodology created by Kent Beck, and this section is based on his book \textit{Extreme Programming Explained (1999)} \citep{xp:explained}. 
Extreme Programming is supposed to be a lightweight, efficient, and fun approach to developing software.

\noindent The methodology consists of the 12 practices:

\begin{tabularx}{\textwidth}{X X X}
	40-hour Week				 & Coding Standards & Collective Ownership \\
	Continuous Integration	  & Metaphor         	 & On-site Customer     \\
	Pair Programming			& Planning Game		& Refactoring          \\
	Simple Design          		  & Small Releases   	& Testing             
\end{tabularx}

Some of these practices are more coherent than others. 
While some enforces each other, others may be unrelated. (i.e. coding standards enforces collective ownership, but on-site customer and pair programming are unrelated.)

\subsection{40-hour Week}
This practice is created to ensure the developers are well rested and fresh when they come in to work.
Although this practice is called 40-hour Week, it should not necessarily be specified to 40-hour per week.
The correct week length will depend on the specific team.

This practice is basis for many of the other practices, as one needs to be well rested to think straight and be creative.
On the other hand, for this practice to work it is required that there are enough tasks to solve.
These tasks are created through the Planning Game practice, hence making it a prerequisite for the 40-hour Week practice to work. 

\subsection{Coding Standards}

\subsection{Collective Ownership}

\subsection{Continuous Integration}

\subsection{Metaphor}

\subsection{On-site Customer}

\subsection{Pair Programming}

\subsection{Planning}

\subsection{Refactoring}

\subsection{Simple Design}

\subsection{Small Releases}

\subsection{Testing}













When looking closer at these practices, it becomes clear that these practices all are based on the principles of \textit{Communication, Courage, Feedback, and Simplicity}.