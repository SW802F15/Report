\section{Extreme Programming explained}
% 2 XP's: 2000 and 200? - differences.
% General introduction to XP2k
Extreme Programming (XP) is a software development methodology created by Kent Beck, and this section is based on his book \textit{Extreme Programming Explained (1999)} \citep{xp:explained}. 
Extreme Programming is supposed to be a lightweight, efficient, and fun approach to developing software.

\noindent The methodology consists of the 12 practices:

\begin{tabularx}{\textwidth}{X X X}
	40-hour Week				 & Coding Standards & Collective Ownership \\
	Continuous Integration	  & Metaphor         	 & On-site Customer     \\
	Pair Programming			& Planning Game		& Refactoring          \\
	Simple Design          		  & Small Releases   	& Testing             
\end{tabularx}

Some of these practices are more coherent than others. 
While some enforces each other, others may be unrelated. (i.e. coding standards enforces collective ownership, but on-site customer and testing are unrelated.)
The correlations can be split into the three categories, Programming, Team Practice, Process.

\subsection{Programming}
The programming category is for the practices directly involved with the coding.
In this category are the practices: Coding Standards, Refactoring, Simple Design, and Testing.

\paragraph{The testing practice} covers both user tests and test-first development.
This means that before writing any code, a unit test specifying the functionality is written.
This does also apply when expanding existing code, meaning before writing the additional code, a unit test is written.

\paragraph{Coding Standards} are important in programming as well as on a team level.
Coding standards at the programming level are used to, quickly letting anyone read code written by others.
Hence making it faster to implement additional features to existing code.

\paragraph{Simple Design} 

\paragraph{Refactoring} is used when the method 






























\subsection{40-hour Week}

\subsection{Coding Standards}

\subsection{Collective Ownership}

\subsection{Continuous Integration}

\subsection{Metaphor}

\subsection{On-site Customer}

\subsection{Pair Programming}

\subsection{Planning}

\subsection{Refactoring}

\subsection{Simple Design}

\subsection{Small Releases}

\subsection{Testing}













When looking closer at these practices, it becomes clear that these practices all are based on the principles of \textit{Communication, Courage, Feedback, and Simplicity}.