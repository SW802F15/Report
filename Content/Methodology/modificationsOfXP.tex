\section{Adaptation of Extreme Programming}
This project is developed with an education purpose, to which some requirements and restrictions exist.
Partly due to these requirements and restrictions, we made some adaptations to XP for it to fit inside the project parameters.

\subsection{40-hour Week}
While we are working on this project, we also follow a range of courses as part of our education.
As a result of this, our schedule will be fragmented when counting project time.
We will therefore not be able to use the literal implementation of this practice.

The purpose of this practice was to ensure energetic developers everyday, and not to overwork them to the point of burnout.
We will implement this purpose by timeboxing the project time.
This way we will still be able to ensure rested and energetic developers regardless of courses and other non-project matters.

\subsection{Code Standards}
Creating a formal code standard is a time consuming task.
We suppose that creating a formal code standard, when we already have an informal code standard, will be too time consuming compared to the benefits.
The informal code standard have been created and expanded through our previous experiences of working together.

\subsection{Continuous Integration}
According to XP this practice should be automated through the setup of a dedicated build server.
Making this an automated task, should ensure the developers working on the current version.
However, because this project is relatively small, the build time is at most a few minutes.
Further, it should not be of a size where there exists several different versions.
Due to these assumptions, we will manually integrate and build the code on a dedicated branch.

We will do this because it will be too difficult to setup and maintain a build server for a Android project using the Gradle build tool. 

\subsection{Metaphor}


\subsection{On-site Customer}
Since this is a semester project, we will not have an actual customer.
In place of the on-site customer we will acts as our own surrogate customer.
This would mean that we are responsible for the priority and requirements decisions, when we act as the surrogate customer.

\subsection{Pair Programming}


\subsection{Planning Game}


\subsection{Refactoring}


\subsection{Simple Design}


\subsection{Small Releases}


\subsection{Testing}




The practices 
Collective Ownership, 
XXXX, 
xXXX
was adapted as they were described by \cite{Beck}.