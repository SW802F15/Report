\Alexander{Check and Correct used time - past-tense vs. future-tense vs. etc}
\section{Extreme Programming}
XP (Extreme Programming) is an agile development methodology. This project will be developed while using the principles of XP.

\subsection{Motivation}
XP follow 12 principles.\Alexander{Source to Beck Book}

These 12 principles should ease the development of high quality software.
This is partly done by ensuring thorough testing, continuously refactoring, and efficient knowledge sharing.
XP further encourages improving oneself and the teams as a whole.
This is partly done by pair programming, sprint reviews, and collaborative ownership.

For these principles to work there are some criteria there needs to be met, some of those criteria are:
\begin{itemize}
\item The size of an XP team should not exceed ten members.
\item Iterations should not exceed four weeks, but two-three weeks are preferable.
\item The team should be self-organising and should not be controlled by a boss.
\item The team members must be able to embrace change.
\item The team should have their workstations placed in the same room.
\end{itemize}

All the listed criteria is fulfilled by this project.

\subsection{Approach}
According to \citet[p. 53]{xp:explained}, the idea behind these practices is that while one practice in itself is weak, the others can cover that weakness. This creates a synergy effect between the practices. This also means that if one or more of the practices are chosen to be discontinued or modified, careful consideration should be made.

%Continued practices
We have chosen to adopt 8 of the 12 practices. 
The adopted practices in this project are:
\begin{itemize}
\item Small Releases
\item Simple Design
\item Testing
\item Refactoring
\item Pair Programming
\item Collective Ownership
\item Continuous integration
\item 40-hour Work Week
\end{itemize}

\paragraph{Testing} will be implemented by writing unit tests and acceptance tests before writing any production code. After the tests and production code is written, they will be run frequently to ensure that everything is working as it is specified, especially after adding new features to the system.

\paragraph{Refactoring} will be implemented by team members making changes where they are needed. Also, code standards will be discussed and adhered to. Pair Programing also aids refactoring since two people programming together will be more likely to have the courage to refactor difficult pieces of code. The Testing practice also aids in this since tests can be run after refactoring which lessens the possibility of the code breaking.

\paragraph{Small Releases} %have been implemented by releasing new, working features with small increments, every iteration consisting of around two weeks.

\paragraph{Simple Design}%\Christoffer{What should we do about this?}

%Discontinued practices
The practices that have been discontinued are:
\begin{itemize}
\item The Planning Game
\item Metaphor
\item On-site Costumer
\item Coding Standards
\end{itemize}

\paragraph{Planning Game} has been discontinued in favour of Planning Poker as described by \citet{xp:planningPoker}.\Christoffer{do we follow this completely?}
The main difference is how conflicting estimations are resolved.
Planning Poker starts with the team discussing the task. This ensures everybody understand the scope and task at hand. This will in-turn reduce disparity of the individual estimates, and make it easier to agree on an estimate. Each member then considers his estimate and keeps it to himself. When all are ready, everybody reveals their estimate at the same time. If there is great disparity between estimates, a discussion is organised. When this discussion is over, everybody estimates the task again. If the conflict still exists, the estimate is decided by ``Optimism wins''.
\Alexander{Find where the source concludes this.}

The Planning Game starts with the team discussing the task. Then each member considers his estimate and reports this to the team. If a conflict have aroused, estimation are chosen by ``Optimism wins'' as stated by \citet[p. 58]{xp:planning}.
This way of reporting individual estimates may influence the estimation of other team member. 
 \Alexander{Explained p.153 description of individual version of planning game (should be used?).
			Planning p.58 description of collaborative version of planning game (used in report).
			We originally chose Poker Planning because we only knew the individual version of Planning Game.}
 \Alexander{Explained p.157, Planning game not necessary four teams of 3-4 developers.}
 

\paragraph{Metaphors} have not been implemented, since we find the product simple enough to understand without any metaphors. 

\paragraph{On-site Costumer} have not been possible to implement, since we do not have a costumer. We have therefore decided to act as our own on-site costumer.
\Alexander{Should be re-written!}

\paragraph{Coding Standards} have not been implemented. Although general guidelines are adhered to.
\Alexander{Should be elaborated upon.}