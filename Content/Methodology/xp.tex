% intro
% %
\section{Extreme Programming}
There is a lot of interest development methodologies in recent years \Kristian{find kilde, etv. omskriv}. One such methodology is XP (Extreme Programming) which we have chosen as a focus of the project, in this section we will look at our motivation for choosing XP, what we hope to accomplice and how we plan to do it.
	
\subsection{Motivation}
What characterises a student project is generally loosely specified objectives, no strict requirements exists, requirements may change doing the project, a small and a project group size of three to six members. These characteristics fit well with the agile development methods. 

Not every characteristics fit well though such the experience of programmers.
%  to learn to program in another way (the extreme way)
% keep track of progress
% create a better product (higher quality)
% be able to respond to change in requirements faster
% * why XP and not another agile method?
% 	XP adv: group size fits, iteration length, self-management
\subsection{Goals}
By using XP for this project we hope to achieve a number of goals:
\begin{itemize}
\item Achieving high test coverage.
\item Produce robust, high quality code.
\item Improve communication.
\item Acquire new skills in agile development.
\end{itemize}
% better code/product quality
% 	* thorough testing
% 	* robustness (compare to earlier projects)
% improved communication in the group
% 	better tracking of issues

When setting goals for the project it is also important to measure how well we achieve these goals. 
\paragraph{Testing Metrics}
When evaluating how well tested a software there is a number of methods available such as line coverage, branch coverage and ...
\Kristian{Vi skal baser dette på Test og Verification kurset}

\paragraph{High Quality Code}
There are many ways to define high quality code, such as lack of bad smells, good readability and simplicity etc...


%Resten er svært at vurdere... 

%\paragraph{Communication}
%How to evaluate improvements in communication is going to be subjective evaluation.

%

\subsection{Approach}
\subsubsection{The 12 Practices}
%which of the 12 practices do we implement?
% 	test-driven development	
%   which do we actually use?