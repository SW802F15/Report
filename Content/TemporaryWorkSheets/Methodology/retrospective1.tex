\subsection*{Retrospective 1}
\paragraph{12. feb - 20. feb}

\subsubsection{Coding Standard}
We have tried to make the code readable by using field-standards (e.g. _privateVariable). Currently some variable names are too short / undescriptive.
We handle standard conflicts when they are discovered.
However a general coding standard has been discussed and decided upon. It was decided that it was not needed to make a specific code standard document.

The coding standard contains decisions about brackets.

\subsubsection{Metaphor}
Not really relevant for the project because the only people looking at the code are us (the developers) and the supervisor and censor (software people).
We instinctively name the classes and methods based on their functionality, so no problems have yet appeared.

\subsubsection{Refactoring}
We have not at this time had the need or reason to refactor.

\subsubsection{Simple Design}
We have very complicated test cases with much replicated code. This is obviously a bad thing, but we contribute this to our inexperience with testing in Android Studio.
We have not prioritised this practise, but we found that the production code (not test) was simple and without replication.

\subsubsection{Pair Programming}
We should limit ourselves to using one computer/monitor and stop using teamviewer. Optimally getting an external monitor to put between us as well as a keyboard and mouse.
We have to change partners more often than we did so far.
We will be better to do all work in pairs, as we until now have solved trivial problem individually.
We have found it difficult to solve some problems in pairs, as pair programming assumes the pair knows about/what to program. We have in these cases assigned the problem to one person (e.g. GUI - no one had any knowledge about the problem, so no matter how long the pair would discuss (about nothing), no solution would be viable. Further the problem of discussing thing you don’t know about.)

\subsubsection{Collective Ownership}
We have worked with ‘collective ownership-like’ approaches before, in that the entire group is responsible for all the code at the exam.
Before we however used the practise of not editing (or reading) code written by other group members.
We will in future sprints incorporate reading the code produced by others and review or refactor it if necessary.

\subsubsection{Testing}
Sometimes we forget to test first. This is bad - we should be more aware of testing first.
We should be better at using setup and teardown.
We have to focus on having the tests act as a specification. We should not put much effort into “test-to-fail”. If it is trivial (boundaries of int, string, etc.) it is okay, but we have to let the tests drive our development rather than hinder it.

\subsubsection{Continuous Integration}
Every time we merge with master we should run all tests and make sure everything runs. Also run the app and make sure it does not crash or have other issues.
We have in this sprint solved some assignments right after another without pushing to master or creating new branches (Play/Next/Prev all in one branch because it was easy). We shall be better to push to master when an assignment is solved and create a new branch for the next assignment.
