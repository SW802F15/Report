\section{Usage Scenario}
A typical usage scenario of the Tempo Player could look as such

\begin{enumerate}
\item The user starts music from the main screen, turns of the screen, and puts the smartphone in a casing, which is located on the user's arm. \Alexander{Better word than ``casing''}
\item The user starts walking at a brisk pace.
\item After a while, the non-graphical user interface (NGUI) is used to change to the next song, by tapping twice on the screen.
\subitem Since the application has now calculated the SPM, the next song has a BPM value which fits the user's pace.
\item The user speeds up, and starts jogging at a slow pace. \Alexander{Better word than ``speeds up'' maybe ``accelerates''}
\item The song finishes, and the application automatically plays the next song which fits the current pace.
\item The user accelerates, and is now running. The screen is tapped twice, and faster music, which fits the new pace, is played.
\item Finally, the user gradually slows down and eventually stops moving completely. A long press (500 ms) is performed on the screen, and the music stops. The exercise session is over. \Alexander{If we have a stop action, then it should be mentioned in introduction and everywhere else.}
\end{enumerate}